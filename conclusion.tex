\chapter{Concluding remarks}

\section{Shearlets}

Shearlets remain an active area of research in image compression, and in spite of the setbacks we detailed in
Chapter~\ref{chap:shearlets}, they are still a relevant possibility for solving transport-type problems,
although a working method is likely to look quite different from the setup we pursued.  Work is currently
being done in this direction which is likely to have more success than ours (for example \cite{ObermeierXX}
using a Fourier space construction of ridgelets), although it is currently in progress.

\section{The Boltzmann equation}

Aside from crude point-wise discretization methods such as the lattice Boltzmann method, the full-grid Fourier
discretization is the {\em de facto} standard for solving both the homogeneous and inhomogeneous Boltzmann
equation. We have presented two other methods. The hyperbolic cross method can outperform full-grid Fourier in
cases where the solution is far from equilibrium, and the polar Laguerre method will be superior in
near-equilibrium cases (in particular with zero-moment solutions). The latter is also fully conservative in
the case of $\beta=1$, unlike any other comparable method (Corollary~\ref{cor:polobs}), and it also requires
no artificial truncation of the collision operator, which makes it free of aliasing. Its successful
implementation and use does not require the choice of parameters such as $R$ or $L$. It also appears to
converge in time, unlike any other existing comparable method, although this is not yet shown. For these
reasons it can be expected to produce solutions of higher quality than other extant methods. This also appears
to be backed up by numerical experiments.

It remains to generalize the polar Laguerre method to higher dimensions. For three dimensions, this could be
accomplished using rotation matrices for spherical harmonics, as discussed in Remark~\ref{rem:rot-pol}.
