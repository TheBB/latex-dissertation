\chapter{Code listings for shearlet library}

The following code listings are mostly real MATLAB code equivalents for the algorithms defined in Section
\ref{sec:algorithms}. There is generally a one-to-one correspondence between algorithm and file.

\section{Parameters}

All parameters for the method as a whole are stored in a persistent structure which can be retrieved by
calling a function {\tt Settings}. Due to issues with parallelization in MATLAB, changes to such persistent
objects are not visible from other threads, so it is advisable to explicitly pass these parameters through
calling signatures whenever possible. Most functions accept an optional last parameter {\tt params}.

%\lstinputlisting{code/shearlets/Settings.m}

\section{Making and manipulating shearlets}

We give here the {\tt GetTransform} method, which gives the matrix and vector for the affine transformation
corresponding to a shearlet, and which implements Algorithm \ref{alg:trf}. The {\tt GetShearletCorners} method
uses {\tt GetTransform} to compute the vertices of the support of a shearlet, one of its subdomains, or the
computational domain as a whole. It corresponds to Algorithm \ref{alg:corners}.

%\lstinputlisting{code/shearlets/GetShearlet.m}
%\lstinputlisting{code/shearlets/Right.m}
%\lstinputlisting{code/shearlets/Up.m}
%\lstinputlisting{code/shearlets/Flip.m}
%\lstinputlisting{code/shearlets/RefreshShearletData.m}
\lstinputlisting[caption={\tt GetTransform}, style=matlab]{code/shearlets/GetTransform.m}
\lstinputlisting[caption={\tt GetShearletCorners}, style=matlab]{code/shearlets/GetShearletCorners.m}

\section{Evaluating shearlets} \label{sec:lst-eval-shlets}

The {\tt PreparePoints} method takes care of all the transforming being done, as well as some work related to
periodicity. It returns the transformed points as well as a vector of boolean values denoting whether or not
this point falls inside the support (which is not always obvious in the periodic case).

%Finally, {\tt EvaluateShearlets} and {\tt EvaluateShearletGradients} provide the same methodology for more
%than one shearlet (unfortunately not vectorized), and {\tt EvaluateLinCombShearlets} and {\tt
%EvaluateLinCombShearletGradients} complete the suite by evaluating linear combinations.

\lstinputlisting[caption={\tt PreparePoints}, style=matlab]{code/shearlets/PreparePoints.m}
%\lstinputlisting{code/shearlets/EvaluateShearlet.m}
%\lstinputlisting{code/shearlets/EvaluateShearletGradient.m}
%\lstinputlisting{code/shearlets/EvaluateShearlets.m}
%\lstinputlisting{code/shearlets/EvaluateShearletGradients.m}
%\lstinputlisting{code/shearlets/EvaluateLinCombShearlets.m}
%\lstinputlisting{code/shearlets/EvaluateLinCombShearletGradients.m}

%\section{Linear and bilinear forms}

%These methods evaluate the integrand of some linear and bilinear forms. They take one or two shearlets and the
%points of evaluation as parameters. We have {\tt BLFKK}, {\tt BLFSS}, {\tt BLFSK} and {\tt BLF} for the
%bilinear forms for the reaction term, the transport term, the cross term, and everything together
%respectively. For linear forms we have similarly {\tt LFK} and {\tt LFS}.

%\lstinputlisting{code/shearlets/BLFKK.m}
%\lstinputlisting{code/shearlets/BLFSS.m}
%\lstinputlisting{code/shearlets/BLFSK.m}
%\lstinputlisting{code/shearlets/BLF.m}
%\lstinputlisting{code/shearlets/LFK.m}
%\lstinputlisting{code/shearlets/LFS.m}

\section{Intersections}

Here, we have code for {\tt CheckPolygonIntersection}, which checks if two convex polygons are disjoint or
not (corresponding to Algorithm \ref{alg:checkintersection}, as well as {\tt CheckShearletIntersection}, which
checks if the supports of two shearlets are disjoint or not (which is not trivial in the period case).

\lstinputlisting[caption={\tt CheckPolygonIntersection}, style=matlab]{code/shearlets/CheckPolygonIntersection.m}
\lstinputlisting[caption={\tt CheckShearletIntersection}, style=matlab]{code/shearlets/CheckShearletIntersection.m}
%\lstinputlisting{code/shearlets/GetIntersection.m}

\section{Building quadrature rules}

We give code for {\tt BuildQuadRuleOnPolygon}, which implements the subdivision routine from
Algorithm~\ref{alg:splitpolygon}. This method is used by {\tt BuildDoubleQuadRule} and {\tt
BuildSingleQuadRule}, which correspond to Algorithms \ref{alg:builddquadrule} and \ref{alg:buildsquadrule}
respectively.

%\lstinputlisting{code/shearlets/SplitTriangle.m}
\lstinputlisting[caption={\tt BuildQuadRuleOnPolygon}, style=matlab]{code/shearlets/BuildQuadRuleOnPolygon.m}
\lstinputlisting[caption={\tt BuildDoubleQuadRule}, style=matlab]{code/shearlets/BuildDoubleQuadRule.m}
\lstinputlisting[caption={\tt BuildSingleQuadRule}, style=matlab]{code/shearlets/BuildSingleQuadRule.m}

%\section{Tying it all together}

%The methods {\tt IntegrateBLF} and {\tt EvaluateBLF} (along with their {\tt LF} namesakes for linear forms)
%take care of the actual integration (using a quadrature rule) in the former case, and orchestrating the whole
%affair of checking intersections and building quadrature rules (and {\em then} integrating) in the second
%case.

%\lstinputlisting{code/shearlets/IntegrateBLF.m}
%\lstinputlisting{code/shearlets/IntegrateLF.m}
%\lstinputlisting{code/shearlets/EvaluateBLF.m}
%\lstinputlisting{code/shearlets/EvaluateLF.m}

%\section{Matrices and vectors}

%On the top of the food chain, we find the functions {\tt Stiffness} and {\tt Load}, which store the stiffness
%matrices and load vectors in persistent variables.  This is also where the parallellization takes place.

%{\tt Stiffness} takes as arguments a matrix identifier, two index sets denoting a submatrix of the global
%stiffness matrix, a bilinear form and a boolean denoting whether or not the matrix should be symmetric. It
%then looks up the matrix in memory, determines which (if any) new elements need to be computed, computes them,
%and then returns the desired submatrix. {\tt Load} works in a similar manner.

%\lstinputlisting{code/shearlets/Stiffness.m}
%\lstinputlisting{code/shearlets/Load.m}

%\subsection{Visualization}

%These methods are helpful for plotting shearlets. They are given here for completeness. {\tt
%PlotSingleShearlet} and {\tt PlotSurroundings} are meant to be called from the other methods, and not by
%themselves.

%\lstinputlisting{code/shearlets/PlotSurroundings.m}
%\lstinputlisting{code/shearlets/PlotSingleShearlet.m}
%\lstinputlisting{code/shearlets/PlotShears.m}
%\lstinputlisting{code/shearlets/PlotShearlets.m}
%\lstinputlisting{code/shearlets/PlotQuadRule.m}
%\lstinputlisting{code/shearlets/PlotPolygon.m}
%\lstinputlisting{code/shearlets/PlotAllShearlets.m}
%\lstinputlisting{code/shearlets/PlotShearlets.m}
%\lstinputlisting{code/shearlets/SurfShearlet.m}
%\lstinputlisting{code/shearlets/QuiverShearletGradient.m}
