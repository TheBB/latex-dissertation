\chapter*{Abstract}

The numerical solution of high-dimensional kinetic transport equations, where the solution is a function on
phase space, is subject to the ``curse of dimensionality'' whereby the rate of convergence becomes too slow to
be practical. This problem can be partly alleviated through sparse tensor discretization, which in turn
requires the formulation of efficient multilevel bases in both physical and velocity space. In this
dissertation, we have investigated some discretization methods for the stationary reaction-advection equation
(physical space) and the spatially homogeneous Boltzmann equation (velocity space).

Shearlets are a system of functions developed for image compression and edge detection, which can efficiently
resolve anisotropic features. We provide some initial considerations in the hopes of adapting shearlets for
solving convection-diffusion problems with constant velocity.

For the spatially homogeneous Boltzmann equation, we study a sparse version of the {\em de facto} standard
Fourier discretization (the hyperbolic cross), which can greatly increase efficiency for solutions far from
equilibrium. We provide analogues to several established results for the theory of the standard Fourier
discretization. We also develop a novel polar discretization based on Laguerre polynomials which, while
generally more expensive than the Fourier method, is considerably more efficient in near-equilibrium
situations and for long times. Unlike the Fourier method it can also be made fully conservative and requires
no truncation of the collision operator.
