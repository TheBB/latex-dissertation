\chapter*{Preface}

\vspace{-0.1cm}
\begin{quoting}
    {\em ``Information can be obtained by averaging over our ignorance''}
    \begin{flushright}
        \vspace{-0.1cm}
        {\em Carlo Cercigniani}
    \end{flushright}
\end{quoting}
\vspace{-0.1cm}

The document you hold is the distilled information I obtained after four years of averaging over my own
ignorance, and that of others. It represents a minute contribution to the cosmos of human knowledge, which is
itself minute compared to the scope of what we hope to achieve. In the grand view of things it is likely to be
largely insignificant.

To myself, however, it is not. To the candidate, a dissertation represents knowledge gained not only about
mathematics but also about his- or herself. No dissertation has ever be written about the latter, but that
does not make it any less significant. Education is only partially about what the professor can teach you
directly.

I owe thanks to many, including the following, in no very particular order: my girlfriend (who has had to
suffer my mental absence for a good while), my family (who has suffered my {\em physical} absence for even
longer), my supervisor Prof.~Ralf Hiptmair (to whose patience I am grateful) and my co-examiner Prof.~Philipp
Grohs (whose insights lie behind many of the theorems in this document), as well as to external co-examiner
Prof.~Sergej Rjasanow. Also to all the participants in the weekly radiative transfer project---in particular
Prof.~Christoph Schwab, whose mathematical vision is as keen as it is ambitious, but also to the other
students who had to bear many discussions which doubtless had no measurable impact on their own projects. I
would also like to thank my colleagues here in Zurich as well as friends in Zurich, Trondheim and
elsewhere (if you feel left out at this point, but still consider yourself my friend, that will have to
suffice).

\begin{flushright}{\em
    Eivind Fonn,
    2013-10-03,
    Zurich, Switzerland.
}\end{flushright}
