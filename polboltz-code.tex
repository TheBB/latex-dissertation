\chapter{Code listings for polar Boltzmann}

The following listings are in the C++ programming language (although mostly pure C), and show some
implementation details from the polar Boltzmann method, both for the assembly of the collision operator and
timestepping. All code here assumes Maxwellian collision kernels and $\beta=1$ or $\beta=2$.

\section{Naming conventions}

Throughout, the following conventions are used.
\begin{itemize}
\item {\tt k}, {\tt l} and {\tt q} denote the indices of a basis function according to \eqref{eqn:pol-enum}.
\item The function space size is given by {\tt Nk} and {\tt Nl}. We also give code for extending an already
computed tensor, in which case {\tt Nk} and {\tt Nl} define the {\em new} size, and {\tt oldNk} and {\tt
oldNl} the {\em old} size. Note that the code makes full use of the fact that {\tt Nl} is assumed to be even!
\item {\tt cdouble} is a synonym for {\tt complex<double>}. Moreover, {\tt stquad} is a synonym for {\tt
structure quad}, which contains quadrature information (definition not shown). Variables of type {\tt stquad}
will invariably be called {\tt quad}.
\item If the preprocessor macro {\tt BASIS2} is defined, $\beta=2$ is assumed, otherwise $\beta=1$.
\item The discrete collision operator is stored in a single-indexed array {\tt tensor}. Indexing is handled by
the functions {\tt tensor\_index} and {\tt tensor\_index\_rev} (not shown).
\item Coefficient vectors are also stored in single-indexed arrays, whose indexing is handled by {\tt
vector\_index} and {\tt vector\_index\_rev} (also not shown).
\end{itemize}

\section{Evaluating basis functions}

The function {\tt evaluate\_bfun} accepts an array of points {\tt pts}, and evaluates the basis function at
that point, storing the results in the array {\tt out}. It also requires two arrays for temporary values,
needed by the recursive evaluation of Laguerre polynomials in the subroutines {\tt laguerre\_evaluate\_zero}
for $L_k^{(0)}$ and {\tt laguerre\_evaluate\_one} for $L_k^{(1)}$. These subroutines are not shown.

If {\tt apply\_weight} is false, this function will return the basis function sans exponential weight, which
is useful in some circumstances.

\lstinputlisting[caption={\tt evaluate\_bfun}, style=c]{code/polboltz/eval.cpp}

\section{Quadrature rules}

Several parts of the code rely on using quadrature rules that are tweaked to precisely the kind of functions
we will be integrating (exponentially weighted polynomials). We reproduce here the quadrature formulas for
the sake of completeness.

\lstinputlisting[caption=Quadrature, style=c]{code/polboltz/quad.cpp}

\section{Evaluating inner integrals}

The function {\tt inner} evaluates the integral part of \eqref{eqn:trf-inner}. It accepts $\|\Bv-\Bv_\ast\|$
and $\|\Bv+\Bv_\ast\|$ as the arguments {\tt vdiff} and {\tt vsum}.

\lstinputlisting[caption={\tt inner}, style=c]{code/polboltz/inner.cpp}

The function {\tt full\_inner} evaluates $\cI^+_{k,l}(\Bv,\Bv_\ast)$ for given $k$ and $l$, at all points
$(\Bv,\Bv_\ast)$ in the squared quadrature rule, and stores them in the double-indexed array {\tt inner\_mx}.
It uses a C++ stdlib {\tt map} with fuzzy floating point comparison to tabulate all the currently evaluated
integrals as functions of $\|\Bv-\Bv_\ast\|$ and $\|\Bv+\Bv_\ast\|$. When it finds a previously computed
match, it uses that instead of reevaluating the whole integral.

\lstinputlisting[caption={\tt full\_inner}, style=c]{code/polboltz/full_inner.cpp}

The definition of the {\tt stfuzzy} structure is given in the following listing. With a suitably large value
of {\tt FUZZY} (we have used $10^6$), and not extremely large quadrature rules, the algorithm will only yield
true matches. This typically speeds up the {\tt full\_inner} function by several orders of magnitude.

\lstinputlisting[caption=Fuzzy floating point comparison, style=c]{code/polboltz/fuzzy.cpp}

\section{Evaluating the discrete collision operator}

The function {\tt outer} evaluates one entry of the gain collision operator $S^{(+)}$. It requires a matrix of
inner integrals computed by {\tt full\_inner}, implicitly depending on $k$ and $l$. It also requires temporary
arrays {\tt left} and {\tt right} to store integrands.

\lstinputlisting[caption={\tt outer}, style=c]{code/polboltz/outer.cpp}

For given values of $k$ and $l$, the function {\tt full\_outer} computes all entries of $S^{(+)}$ that
requires the computation of $\cI^{+}_{k,l}$. This ensures that there is no overlap in calling {\tt
full\_inner} from different CPUs. The computed values are stored in the single-indexed array {\tt tensor}.
This function also computes only values which are needed, as specified by the arguments {\tt oldNk} and {\tt
oldNl}.

\lstinputlisting[caption={\tt full\_outer}, style=c]{code/polboltz/full_outer.cpp}

Finally, the {\tt add\_loss} function is responsible for adding the loss term.

\lstinputlisting[caption={\tt add\_loss}, style=c]{code/polboltz/add_loss.cpp}

In {\tt compute} we illustrate how one can use OpenMP to run such a computation. The job allocation is stored
in a struct {\tt stjobqueue jall} which is not shown.

\lstinputlisting[caption={\tt compute}, style=c]{code/polboltz/compute.cpp}

\section{Timestepping}

For applying the collision tensor to a coefficient vector {\tt vec}, storing the results in the {\tt out}
array, we use the {\tt collide} function, which works much like {\tt compute}. It also supports vector sizes
which may be different from the tensor sizes, as given by {\tt tensNk} and {\tt tensNl}. The quantity {\tt
aNl} defines the set of active coefficients, and may be smaller than {\tt tensNl}.

Since this is a fairly quick routine, we have used a job allocator instead of a job queue.

\lstinputlisting[caption={\tt collide}, style=c]{code/polboltz/collide.cpp}

The following listing performs time timestepping using a fourth-order explicit Runge-Kutta rule. Note the call
to {\tt adapt\_check}, which is given in the last listing.

\lstinputlisting[caption={\tt rk\_timestep}, style=c]{code/polboltz/rk_timestep.cpp}
\lstinputlisting[caption={\tt adapt\_check}, style=c]{code/polboltz/adapt_check.cpp}
