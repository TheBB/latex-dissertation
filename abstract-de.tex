\chapter*{Kurzfassung}

Das numerische L\"osen hochdimensionaler kinetischer Transportgleichungen, bei denen die L\"osung sowohl von
der Position, als auch von der Geschwindigkeit abh\"angt, unterliegt dem sogenannten \glqq Fluch der
Dimensionalit\"at\grqq, wegen dem die Konvergenzraten klassischer Verfahren zu langsam werden, um n\"utzlich zu
sein. Dieses Problem kann teilweise durch Dünn\-gitter\-diskretisierung ab\-geschw\"acht werden, wof\"ur man
effektive Mehrstufen\-basen f\"ur die Teilr\"aume (Ort und Geschwindigkeit) braucht. In dieser Dissertation
studieren wir Methoden f\"ur die station\"are Reaktions\-advektions\-gleichung und die r\"aumlich homogene
Boltzmann-Gleichung.

\glqq Shearlets\grqq \ sind ein System von Funktionen, das f\"ur Bildkompression und Kantendetektion
entwickelt wurde, welches anisotropische Eigenschaften effektiv erfassen kann. Wir versuchen, dieses f\"ur die
L\"osung von Reaktions\-advektions\-gleichungen mit konstanter Geschwindigkeit anzupassen.

F\"ur die r\"aumlich homogene Boltzmann-Gleichung studieren wir eine d\"unne Version des {\em de facto}
Standards f\"ur Fourierdiskretation (das Hyperbelkreuz), mit dem die Effektivit\"at weit entfernt von
Gleichgewicht erh\"oht werden kann. Wir zeigen analoge Ergebnisse f\"ur einige Theoreme, die bereits f\"ur die
Standard-Fourierdiskretisierung bewiesen wurden. Des Weiteren entwickeln wir eine neue Polar\-diskretisierung,
basierend auf Laguerre-Polynomen, die zwar generell teu\-rer als die Fouriermethode ist, aber daf\"ur
effektiver nahe dem Gleichgewicht bzw.\ f\"ur lange Zeiten. Im Gegensatz zur Fouriermethode kann sie auch
v\"ollig konservativ gemacht werden und ben\"otigt kein Abschneiden des Kollisionoperators.
